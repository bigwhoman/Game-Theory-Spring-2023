\smalltitle{سوال 2}
        \begin{enumerate}
            \item \phantom{text}
            \begin{latin}
                \begin{center}
                    \begin{equation*}
                                \begin{matrix}
                                      & & 1 & 2 & 3 & ...       & n            \\
                                    & 1 & \omega_{11} - W_1 & \omega_{12} & ... &     & \omega_{1n}  & p_1         \\
                                    & 2 & \omega_{21} & \omega_{22} - W_2 & ... &      & \vdots & p_2   \\
                                    & 3 & \omega_{31} & \omega_{32} & ... &      & \vdots & \vdots\\
                                    & \vdots \\
                                    & n & \omega_{n1} & \omega_{n2} & ... & ...  & \omega_{nn} - W_n  & p_n \\
                                    & Min & \omega_{11} - W_1 &  & ... & ...   & q_n 
                                    \end{matrix}
                                \end{equation*}
                           \end{center}
            We know player 1, plays each node with probability $p_i$ and player 2 plays with probability $p_j$ so the expected payoff of player 1 is 
            $\omega_{ij} - W_i \times p_j$. \\
            now we need to find that for each i and j what is the expected payoff : \\
            $\sum_i{W_i * \sum_j[p_j * (\omega_{ij}-W_i*p_j)]}$ thus the expression is written as $\sum_i{W_i * \sum_j{p_j*\omega_{ij}}-W_i^2*\sum_j(p_j^2)} 
            \xrightarrow{p_j \leq 1 \rightarrow p_j^2 \leq 1} \leq \sum_i{W_i * \sum_j{p_j * \omega_{ij}}-W_i^2}$\\ \\
            Now we could simplify the expression and see that : \\
            $\sum_i{W_i * \sum_j{p_j*\omega_{ij}}-W_i^2} = \sum{i}\sum{j}(p_i * p_j * \omega_ij)$\\
            so we would conclude that the expected payoff of player 1 is the expected value of edges and the game is 0 value.
            \end{latin}
            \item \phantom{text}
            \begin{latin}
                  The game is 0 value so there exists an equilibrim that both players payoffs are 0.
                  Now like the first part, think as if $probability_{p1} = X,probability_{p2} = Y$ \\ 
                  $E(P1 = x^TAy) \xrightarrow{min_xmax_yx^TAy=0} max_y = 0 \Rightarrow A^Tx \leq 0 ,0 \leq x, 0 \leq y , \sum_i{x_i} = 1,\sum_j{y_j} = 1$\\
                  Now the above problem is simply an LP with constraints and because that this game is a zero-sum game, the optimal values are the same as first problem.\\
                  Because we found an optimal point, there exists a solution in which $A^Tx=0 \Rightarrow x^TA = 0 $

            \end{latin}
          
        \end{enumerate}